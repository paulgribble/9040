\documentclass[12pt]{article}

% === Preamble: packages ===
\usepackage[margin=1in]{geometry}       % page margins
\usepackage{amsmath}                     % equations
\usepackage{graphicx}                    % figures
\usepackage{booktabs}                    % professional tables
\usepackage{hyperref}                    % clickable links & refs
\usepackage[
  backend=biber,
  style=apa,
  natbib=true
]{biblatex}                              % citations
\addbibresource{refs.bib}

\title{Effects of Caffeine on Grip Strength: A Double-Blind Study}
\author{A.\ Student}
\date{February 2025}

% === Document body ===
\begin{document}
\maketitle

\begin{abstract}
We investigated whether a moderate dose of caffeine (200\,mg)
increases maximal isometric grip strength in healthy young adults
using a double-blind, placebo-controlled, crossover design.
Thirty participants completed grip strength testing before and after
caffeine or placebo ingestion. Caffeine produced a small but
significant increase in grip strength.
\end{abstract}

\section{Introduction}

Caffeine is the most widely consumed psychoactive substance in the
world \citep{fredholm1999actions}. While its cognitive effects are
well-documented, its influence on motor performance---particularly
muscular strength---has received less attention
\citep{kalmar2004caffeine}.

A recent meta-analysis found a small positive effect of caffeine on
muscular strength and endurance \citep{warren2010effect}, but the
mechanisms remain debated. The present study examined the effect of
200\,mg caffeine on maximal isometric grip strength.

\section{Methods}

\subsection{Participants}

We recruited 30 healthy adults (15 female, mean age = 23.1 years,
\textit{SD} = 3.4) from the university community. Inclusion criteria
required participants to be regular caffeine consumers (1--3 cups
of coffee per day) with no history of cardiovascular disease.

\subsection{Procedure}

Participants attended two testing sessions separated by at least 48
hours. At each session, they completed three maximal voluntary
contractions (MVCs) on a hand dynamometer, consumed either caffeine
or placebo, waited 45 minutes, and then completed three more MVCs.

\subsection{Data Analysis}

Change scores (post minus pre) were computed for each session.
We compared caffeine and placebo change scores using a
paired-samples $t$-test.

% ==============================================================
% TASK 1: EQUATION
%
% Add a numbered equation for Pearson's correlation coefficient
% below this comment. Use the equation environment:
%
% \begin{equation}
%   r = \frac{\sum_{i=1}^{n}(x_i - \bar{x})(y_i - \bar{y})}
%            {\sqrt{\sum_{i=1}^{n}(x_i - \bar{x})^2
%             \sum_{i=1}^{n}(y_i - \bar{y})^2}}
%   \label{eq:pearson}
% \end{equation}
%
% Then add a sentence referencing it:
%   "We also computed the correlation ... (Equation~\ref{eq:pearson})."
%
% Uncomment the lines above (remove the % signs) and adjust as needed.
% ==============================================================


\section{Results}

% ==============================================================
% TASK 2: TABLE
%
% Create a table showing the results. Use the template below.
% Uncomment it and fill in the data.
%
% \begin{table}[htbp]
%   \centering
%   \caption{Mean grip strength (N) by condition and time point.
%            Standard deviations in parentheses.}
%   \label{tab:results}
%   \begin{tabular}{lcc}
%     \toprule
%     Condition & Pre & Post \\
%     \midrule
%     Caffeine  & ??? & ??? \\
%     Placebo   & ??? & ??? \\
%     \bottomrule
%   \end{tabular}
% \end{table}
%
% Use these values:
%   Caffeine: Pre = 38.2 (5.1), Post = 40.5 (5.3)
%   Placebo:  Pre = 37.9 (4.8), Post = 38.3 (5.0)
%
% Then reference the table in the text:
%   "Results are summarized in Table~\ref{tab:results}."
% ==============================================================

The caffeine condition produced a mean increase of 2.3\,N
(\textit{SD} = 4.1), compared to 0.4\,N (\textit{SD} = 3.8) in
the placebo condition. This difference was statistically
significant, $t(29) = 2.18$, $p = .038$, $d = 0.40$.

% ==============================================================
% TASK 3: FIGURE
%
% Include the sample figure. Uncomment and complete:
%
% \begin{figure}[htbp]
%   \centering
%   \includegraphics[width=0.7\textwidth]{sample_figure.pdf}
%   \caption{Your caption here. Describe what the figure shows.}
%   \label{fig:grip}
% \end{figure}
%
% Then reference it: "as shown in Figure~\ref{fig:grip}."
%
% NOTE: A sample figure (sample_figure.pdf) is included in this
% folder. If you are on Overleaf, upload it to your project.
% ==============================================================


\section{Discussion}

Our findings indicate a small but significant ergogenic effect of
caffeine on grip strength, consistent with prior work
\citep{warren2010effect}. The effect size ($d = 0.40$) suggests a
practically meaningful, though modest, benefit.

% ==============================================================
% TASK 4: CITATION
%
% Add a new reference to refs.bib and cite it somewhere in the
% Discussion. For example, you might cite a paper on caffeine
% mechanisms and write a sentence like:
%
%   "This may reflect caffeine's action as an adenosine receptor
%    antagonist \citep{your_new_key}."
%
% Remember to add the BibTeX entry to refs.bib!
% ==============================================================

Future research should examine dose--response relationships and
whether these effects extend to dynamic strength tasks such as
vertical jumping or sprinting.

\printbibliography

\end{document}
